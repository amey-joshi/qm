\documentclass{article}
\usepackage{amsmath, amssymb, amsthm, amsfonts}
\numberwithin{equation}{section}
\begin{document}
\title{Variational Methods in Quantum Mechanics}
\author{Amey Joshi}
\maketitle
\section{Introduction}\label{s1}
\section{The variation method}\label{s2}
The variation method relies on the fact that the quantity
\begin{equation}\label{s2e1}
E = \int\phi^\ast\hat{H}\phi dV
\end{equation}
is greater than the ground state energy eigenvalue of the Hamiltonian 
$\hat{H}$. The function $\phi$ is an arbitrary square integrable function
which satisfies the appropriate boundary conditions and therefore can be 
a candidate solution of the Schr\"{o}dinger equation $\hat{H}\phi = E\phi$.
We call $\phi$ the variation function. As the set of eigenfunction of 
$\hat{H}$ form a complete orthonormal set $\{\psi_0, \psi_1, \ldots\}$, 
we can write $\phi$ as
\begin{equation}\label{s2e2}
\phi = \sum_{n \ge 0}a_n\psi_n.
\end{equation}
Substituting equation \eqref{s2e2} into equation \eqref{s2e1}, we get
\[
E = \sum_{m,n \ge 0}a_m^\ast a_n \int \psi_m^\ast \hat{H}\psi_n dV.
\]
If $\hat{H}\psi_n = W_n\psi_n$ then,
\[
E = \sum_{m,n \ge 0}a_m^\ast a_n W_n \int\psi_m^\ast\psi_n dV = 
\sum_{n \ge 0}|a_n|^2W_n.
\]
If $W_0$ is the eigenvalue of the ground state then
\begin{equation}\label{s2e3}
E - W_0 = \sum_{n \ge 0}|a_n|^2(W_n - W_0),
\end{equation}
where we have used the fact that $\sum_{n \ge 0}|a_n|^2 = 1$. Since 
$W_n \ge W_0$ for all $n$, we readily infer that
\begin{equation}\label{s2e4}
E \ge W_0.
\end{equation}
Thus, the energy calculated using equation \eqref{s2e1} is an 
\emph{upper bound} to the ground state eigenvalue of the Hamiltonian 
$\hat{H}$. 

In order to use equation \eqref{s2e4} for finding approximate ground state
eigenfunctions of $\hat{H}$ one normally chooses $\phi$ to depend on some 
parameter, say $\beta$. Thus we choose, $\phi \equiv \phi(\beta, \vec{x})$
so that equation \eqref{s2e4} becomes
\begin{equation}\label{s2e5}
E(\beta) \ge W_0.
\end{equation}
We then choose a $\beta$, say $\beta_0$ that minimizes $E(\beta)$. The 
function $\phi(\beta_0, \vec{x})$ is also the closest to the true ground 
state eigenfunction $\psi$, where the `closeness' is measured as the usual 
$L^2$ norm.

Now suppose that we know the eigenvalues of the ground state and the first 
excited state of the system. We can use them to find out how close the 
function $\phi$ that minimizes $E - W_0$ is to $\psi_0$, the true but 
unknown ground state eigenfunction. We mentioned that a measure of 
closeness between the functions is the $L^2$ norm of their differences. 
Let us therefore define
\begin{equation}\label{s2e6}
\epsilon = \int (\phi - \psi_0)^\ast(\phi - \psi)dV.
\end{equation}
If we assume that both $\phi$ and $\psi_0$ are normalised then
\[
\epsilon = 2 - \int\phi^\ast\psi_0 dV - \int\phi\psi_0^\ast dV
\]
Using equation \eqref{s2e2},
\[
\epsilon = 2 - \sum_{n \ge 0}a_n^\ast\int\psi_n^\ast\psi_0 dV - \sum_{n \ge 0}a_n\int\psi_n\psi_0^\ast dV.
\]
Using the orthonormality of the set $\{\psi_0, \psi_1, \ldots\}$, we get
\begin{equation}\label{s2e7}
\epsilon = 2(1 - \Re(a_0)),
\end{equation}
where $\Re(\cdot)$ denotes the real value of a complex number. 

We can write equation \eqref{s2e3} as
\[
E - W_0 = \sum_{n \ge 0}|a_n|^2(W_n - W_0) \ge (W_1 - W_0)\sum_{n \ge 1}|a_n|^2.
\]
Since $\sum_{n \ge 0}|a_n|^2 = 1$, we can as well write
\begin{equation}\label{s2e8}
E - W_0 \ge (W_1 - W_0)(1 - |a_0|^2).
\end{equation}

When $\epsilon \ll 1$, equation \eqref{s2e7} suggests that $\Re(a_0)$ is very
close to $1$. Therefore, $1 + \Re(a_0) \approx 1 + 1 = 2$ and therefore,
\begin{equation}\label{s2e9}
\epsilon \approx (1 - \Re^2(a_0)).
\end{equation}
Further, since $\sum_{n \ge 0}|a_n|^2 = 1$, if $\Re(a_0)$ is very close to $1$
then $\Im(a_0)$ must be very close to zero. Therefore, $|a_0|^2 \approx
\Re(a_0)^2$, allowing us to write \eqref{s2e8} as
\begin{equation}\label{s2e10}
E - W_0 \ge (W_1 - W_0)(1 - \Re^2(a_0)).
\end{equation}
Combining it with equation \eqref{s2e9} we get
\begin{equation}\label{s2e11}
\frac{E - W_0}{W_1 - W_0} \ge \epsilon.
\end{equation}
We can use \eqref{s2e7} to get 
\begin{equation}\label{s2e12}
1 - \Re(a_0) \le \frac{1}{2}\frac{E - W_0}{W_1 - W_0}.
\end{equation}
$W_0$ and $W_1$ are the experimentally determined quantities while $E$ is
calculated from our choice of $\phi$. Equation \eqref{s2e12} tells how far
does $\Re(a_0)$ deviate from $1$ given these quantities. The closer $\Re(a_0)$
is to $1$, the better is $\phi$ as an approximation to $\psi_0$.

\section{Examples of using variation method}\label{s3}
\subsection{Helium-like atoms}
Consider an atom like He or ions like Li$^{+}$ or Be$^{2+}$. They all have
two electrons and $Z$ protons, where $Z = 2, 3, 4$ respectively for He, Li$^{+}$,
Be$^{2+}$. The Hamiltonian, in cgs units, is
\begin{equation}\label{s3e1}
\hat{H} = -\frac{\hslash^2}{2m}\left(\nabla_1^2 + \nabla_2^2\right)\psi(\vec{x}_1, \vec{x}_2)
- \frac{Ze^2}{r_1} - \frac{Ze^2}{r_2} + \frac{e^2}{r_{12}}.
\end{equation}
Here $\vec{x}_1$ and $\vec{x}_2$ are the position vectors of the two electrons
with respect to the nucleus at the origin, $r_1 = |\vec{x}_1|, r_2 = |\vec{x}_2|,
r_{12} = |\vec{x}_1 - \vec{x}_2|$, $e$ is the electronic charge, $m$ the 
electronic mass, $\nabla_1^2$ is the laplacian in terms of $\vec{x}_1$ coordinates 
and $\nabla_2^2$ is that in terms of $\vec{x}_2$ coordinates. We consider a
trial function
\begin{equation}\label{s3e2}
\phi(\vec{x}_1, \vec{x}_2) = \phi_1(\vec{x}_1)\phi_2(\vec{x}_2),
\end{equation}
where
\begin{eqnarray}
\phi_1(\vec{x}_1) &=& \left(\frac{Z^{\prime 3}}{\pi a_0^3}\right)^{1/2}\exp\left(-\frac{Z^\prime r_1}{a_0}\right) \\
\phi_2(\vec{x}_2) &=& \left(\frac{Z^{\prime 3}}{\pi a_0^3}\right)^{1/2}\exp\left(-\frac{Z^\prime r_2}{a_0}\right)
\end{eqnarray}
$Z^\prime$ is the `effective atomic number', a free parameter in our variational
analysis and 
\begin{equation}\label{s3e5}
a_0 = \frac{h^2}{4\pi^2 me^4},
\end{equation}
is the `radius of the first orbit' of the Hydrogen atom. The functions $\phi_1$ 
and $\phi_2$ are eigenfunctions of the hydrogenic Hamiltonians with atomic
number $Z^\prime e$. That is,
\begin{eqnarray}
-\frac{\hslash^2}{2m}\nabla_1^2\phi_1 - \frac{Z^\prime e^2}{r_1} &=& -Z^{\prime 2}W_H\phi_1 \label{s3e6} \\
-\frac{\hslash^2}{2m}\nabla_2^2\phi_2 - \frac{Z^\prime e^2}{r_2} &=& -Z^{\prime 2}W_H\phi_2 \label{s3e7}, 
\end{eqnarray}
where 
\begin{equation}\label{s3e8}
W_H = \frac{e^2}{2a_0}.
\end{equation}
Using equations \eqref{s3e1} and \eqref{s3e2} in equation \eqref{s2e1}, we get
\begin{equation}\label{s3e9}
E = \int dV_1 dV_2 \phi_1^\ast\phi_2^\ast\hat{H}\phi_1\phi_2.
\end{equation}
Now, 
\begin{equation}\label{s3e10}
\hat{H}\phi_1\phi_2 = (\hat{H}_1\phi_1)\phi_2 + (\hat{H}_2\phi_2)\phi_1 + (Z^\prime - Z)e^2\left(\frac{1}{r_1} + \frac{1}{r_2}\right)\phi + \frac{e^2}{r_{12}}\phi,
\end{equation}
where $\hat{H}_1$ and $\hat{H}_2$ are hydrogenic Hamiltonians with atomic number
$Z^\prime$. Using equations \eqref{s3e6} and \eqref{s3e7} we get
\begin{equation}\label{s3e11}
\hat{H}\phi_1\phi_2 = -2Z^{\prime 2}W_H\phi + (Z^\prime - Z)e^2\left(\frac{1}{r_1} + \frac{1}{r_2}\right)\phi + \frac{e^2}{r_{12}}\phi.
\end{equation}
Substituting \eqref{s3e11} into \eqref{s3e9} we get
\begin{equation}\label{s3e12}
E = -2Z^{\prime 2}W_H + (Z^\prime - Z)e^2\int dV_1dV_2\phi^\ast\left(\frac{1}{r_1} + \frac{1}{r_2}\right)\phi + \int dV_1dV_2\frac{e^2|\phi|^2}{r_{12}}.
\end{equation}
The integral in the second term can be written as
\[
I_2 = 2\int dV_1\frac{|\phi_1|^2}{r_1} = 8\pi\left(\frac{Z^{\prime 3}}{\pi a_0^3}\right)\int_0^\infty\exp\left(-\frac{2Z^\prime r_1}{a_0}\right)r_1 dr_1.
\]
If $\rho = 2Z^\prime r_1/a_0$ then
\[
I_2 = \frac{2Z^\prime}{a_0}\int_0^\infty e^{-\rho}\rho d\rho = \frac{2Z^\prime}{a_0}.
\]
The second term of equation \eqref{s3e12} can now be written as
\[
(Z^\prime - Z)e^2 \times \frac{2Z^\prime}{a_0} = 4(Z^\prime - Z)Z^\prime W_H,
\]
where $W_H$ is given by \eqref{s3e8}. Thus,
\begin{equation}\label{s3e13}
E = -2Z^{\prime 2}{W_H} + 4(Z^\prime - Z)Z^\prime W_H +  \int dV_1dV_2\frac{e^2|\phi|^2}{r_{12}}.
\end{equation}A
Evaluation of the third term is a little challenging. We proceed as follows. 
The integrand of the term is
\[
\frac{e^2}{r_{12}}\frac{Z^{\prime 6}}{\pi^2 a_0^6}\exp\left(-\frac{2Z^\prime}{a_0}(r_1 + r_2)\right).
\]
and the measure of the integral is $r_1^2\sin\vartheta_1dr_1d\varphi_1 d\vartheta_1 r_2^2\sin\vartheta_2dr_2d\varphi_2 d\vartheta_2$.
Introduce new variables $\rho_1 = -2Z^\prime r_1/a_0$ and $\rho_2 = -2Z^\prime r_2/a_0$. Then
\[
|\vec{x}_1 - \vec{x}_2| = |r_1 \vec{e}_{r_1} - r_2 \vec{e}_{r_2}| = \frac{a_0}{2Z^\prime}|\rho_1\vec{e}_{\rho_1} - \rho_2\vec{e}_{\rho_2}|
\]
so that
\begin{equation}\label{s3e14}
\frac{1}{r_{12}} = \frac{2Z^\prime}{a_0}\frac{1}{|\vec{\rho}_1 - \vec{\rho}_2|} = \frac{2Z^\prime}{a_0}\frac{1}{\rho_{12}}.
\end{equation}
The measure of the integral becomes
\begin{equation}\label{s3e15}
dV_1 dV_2 = \frac{a_0^6}{64 Z^{\prime 6}}\rho_1^2\sin\vartheta_1d\rho_1 d\varphi_1 d\vartheta_1\rho_2^2\sin\vartheta_2d\rho_2 d\varphi_2 d\vartheta_2.
\end{equation}
The third term of equation \eqref{s3e13} thus becomes $T_3 = $
\begin{equation}\label{s3e16}
\int_0^\infty\int_0^\pi\int_0^{2\pi}\int_0^\infty\int_0^\pi\int_0^{2\pi}\frac{Z^\prime e^2}{32\pi^2 a_0}\frac{e^{-\rho_1 - \rho_2}}{\rho_{12}}\rho_1^2\sin\vartheta_1\rho_2^2\sin\vartheta_2 
d\rho_1d\varphi_1d\vartheta_1 d\rho_2d\varphi_2d\vartheta_2
\end{equation}
The angular integrals can be readily evaluated so that
\begin{equation}\label{s3e17}
T_3 = \frac{Z^\prime e^2}{32\pi^2 a_0}\int_0^\infty\int_0^\infty \frac{(4\pi\rho_1^2e^{-\rho_1}) \times (4\pi\rho_2^2e^{-\rho_2})}{\rho_{12}}d\rho_1 d\rho_2.
\end{equation}
The integral, aside from the constant, is the electrostatic energy of two
spherically symmetric charge distributions of densities $e^{-\rho_1}$ and
$e^{-\rho_2}$. We will, therefore, use tricks from electrostatics to evaluate
it. The potential due to a spherical shell with charge density $4\pi\rho_1^2e^{-\rho_1}$
is
\[
dV(r) = \begin{cases}
4\pi\rho_1^2e^{-\rho_1}\frac{1}{\rho_1} & r < \rho_1 \\
4\pi\rho_1^2e^{-\rho_1}\frac{1}{r}      & r \ge \rho_1
\end{cases}
\]
Therefore, the potential at $r$ due to the entire distribution is
\[
V(r) = \frac{4\pi}{r}\int_0^re^{-\rho_1}\rho_1^2d\rho_1 + 4\pi\int_r^\infty e^{-\rho_1}\rho_1d\rho_1 = \frac{4\pi}{r}(2 - e^{-r}(r + 2)).
\]
Equation \eqref{s3e17} now becomes
\[
T_3 = \frac{Z^\prime e^2}{32\pi^2 a_0}\cdot 4\pi\cdot\int_0^\infty V(\rho_2)e^{-\rho_2}\rho_2^2d\rho_2
 = \frac{Z^\prime e^2}{2a_0}\int_0^\infty [2 - e^{-\rho_2}(\rho_2 +2)]e^{-\rho_2}\rho_2d\rho_2.
\]
The integral evaluates to $5/4$ and hence
\begin{equation}\label{s3e18}
T_3 = \frac{5}{4}\frac{Z^\prime e^2}{2a_0} = \frac{5}{4}Z^\prime W_H,
\end{equation}
where we used \eqref{s3e8} for the definition of $W_H$. Substituting this in
equation \eqref{s3e13} we get
\begin{equation}\label{s3e19}
E = \left[-2Z^{\prime 2} + 4Z^\prime(Z^\prime - Z) + \frac{5}{4}Z^\prime\right]W_H
\end{equation}
The variational procedure requires that we minimize $E$ with respect to $Z^\prime$.
Taking the derivative with respect to $Z^\prime$, we get
\[
\frac{\partial E}{\partial Z^\prime} = \left[-4Z^\prime - 4Z + 8Z^\prime + \frac{5}{4}\right]W_H
\]
At the extremum,
\begin{equation}\label{s3e20}
Z^\prime = Z - \frac{5}{16}.
\end{equation}
The minumum value of $E$ is
\begin{equation}\label{s3e21}
E_{\text{min}} = -2\left(Z - \frac{5}{16}\right)^2W_H.
\end{equation}
For Helium aton $Z = 2$. $W_H$ is always $-13.6$ eV. Therefore, $E_{\text{min}} 
= -77.45625$ eV. The actual energy level is $-79.00515$ eV. Our estimate is
$2\%$ off the actual value.

\section{Upper and lower bounds}\label{s4}
Equation \eqref{s2e4} gives an upper bound for the energy eigenvalue of the ground
state. However, to get a good feel of what $E$ is we also need a lower bound. The 
theory to get the lower bound was available since the 1930s through the work of D.
H. Weinstein. We will describe it in this section. Once again let $\phi$ be a
variational function as defined in section \ref{s2}. Define the quantities
\begin{eqnarray}
E &=& \int\phi^\ast\hat{H}\phi dV \label{s4e1} \\
D &=& \int \left(\hat{H}\phi\right)^\ast \hat{H}\phi dV \label{s4e2}
\end{eqnarray}
If we expand $\phi$ in terms of the eigenfunctions $\psi_n$ of $\hat{H}$ as we
did previously in equation \eqref{s2e2}, we get
\begin{eqnarray}
E &=& \sum_{n \ge 0}|a_n|^2W_n \label{s4e3} \\
D &=& \sum_{n \ge 0}|a_n|^2W_n^2 \label{s4e4}
\end{eqnarray}
Let
\begin{equation}\label{s4e5}
\Delta = D - E^2.
\end{equation}
We manipulate the right hand side of the above equation as
\[
\Delta = D - 2E^2 + E^2 = D - 2EE + E^2 \cdot 1 = \sum_{n \ge 0}|a_n|^2W_n^2 
-2E\sum_{n \ge 0}|a_n|^2W_n + E^2\sum_{n \ge 0}|a_n|^2,
\]
where we have used the fact that $\phi$ is normalized so that
\[
\sum_{n \ge o}|a_n|^2 = 1.
\]
We can now write
\begin{equation}\label{s4e6}
\Delta = \sum_{n \ge 0} |a_n|^2(W_n - E)^2.
\end{equation}
Among all the energy levels in the spectrum of $\hat{H}$ there will be one, say
$W_k$ that will be closest to $E$. Then $(W_k - E)^2 \le (W_n - E)^2$. Therefore,
equation \eqref{s4e6} becomes
\[
\Delta \ge (W_k - E)^2.
\]
Of the two possibilities $W_k \ge E$ and $W_k < E$, the first one gives
\begin{equation}\label{s4e7}
E + \sqrt{\Delta} \ge W_k \ge E
\end{equation}
while the second one gives
\begin{equation}\label{s4e8}
E > W_k > E - \sqrt{\Delta}.
\end{equation}
We can combine equations \eqref{s4e7} and \eqref{s4e8} to get
\[
E - \sqrt{\Delta} \le W_k \le E + \sqrt{\Delta}.
\]
Using the definition of $\Delta$ from equation \eqref{s4e5} we get
\begin{equation}\label{s4e9}
E - \sqrt{D - E^2} \le W_k \le E + \sqrt{D - E^2}.
\end{equation}
This equation gives an interval in which the energy level closest to $E$ 
lies.

\section{Examples of bounds}\label{s5}
We now consider a few problems where we calculate the upper and lower
bounds. We begin with problems whose exact solution is known just to
illustrate the method applied to simple problems.
\subsection{The harmonic oscillator}
The Hamiltonian is
\begin{equation}\label{s5e1}
\hat{H} = -\frac{\hslash^2}{2m}\frac{d}{dx^2} + \frac{1}{2}m\omega^2x^2,
\end{equation}
where $m$ is the mass of the particle and it is subject to a restoring
force $F = -m\omega^2 x$. Let us begin with a trial function
\begin{equation}\label{s5e2}
\phi(x;\alpha) = Ne^{-\alpha x^2},
\end{equation}
where $\alpha$ is an unknown constant and $N$ is the normalization constant.
The function $\phi$ is indeed a well-behaved function that can be a wave 
function of a physical system. The normalization constant can be easily
found to be
\[
N = \left(\frac{2\alpha}{\pi}\right)^{1/4}
\]
so that the normalized trial function is
\begin{equation}\label{s5e3}
\phi(x;\alpha) = \left(\frac{2\alpha}{\pi}\right)^{1/4}e^{-\alpha x^2}.
\end{equation}
We next calculate
\begin{equation}\label{s5e4}
\hat{H}\phi = \left(\frac{2\alpha}{\pi}\right)^{1/4}\left(\frac{\hslash^2
\alpha}{m}e^{-\alpha x^2} + \left(\frac{m\omega^2x^2}{2} - \frac{2\hslash^2
\alpha^2}{m}\right)x^2e^{-\alpha x^2}\right)
\end{equation}
and 
\begin{equation}\label{s5e5}
E(\alpha) = \langle\hat{H}\rangle = \frac{\alpha\hslash^2}{2m} + 
\frac{m\omega^2}{8\alpha}.
\end{equation}
The extreme value of $E$ can be found by setting $E^\prime(\alpha) = 0$. It
occurs at
\begin{equation}\label{s5e6}
\alpha_0 = \pm\frac{m\omega}{2\hslash}.
\end{equation}
The minimum $E$ is
\begin{equation}\label{s5e7}
E = \frac{1}{2}\hslash\omega
\end{equation}
and the function $\phi$ for this choice of $\alpha$ is
\begin{equation}\label{s5e8}
\phi(x;\alpha_0) = \left(\frac{2m\omega}{h}\right)^{1/4}\exp
\left(-\frac{1}{2}\frac{m\omega}{\hslash}x^2\right).
\end{equation}
Let us pretend that we do not know that equation \eqref{s5e7} is indeed
the energy of the ground state. In order to find out how far away we are
from the true ground state energy, we need to compute $\sqrt{D - E^2}$,
where the quantity $D$ is defined by equation \eqref{s4e2}. In order to
calculate $D$, we first note that, for the $\phi$ given by equation
\eqref{s5e8},
\begin{equation}\label{s5e9}
\hat{H}\phi = \frac{1}{2}\hslash\omega
\end{equation}
so that
\begin{equation}\label{s5e10}
D = \frac{1}{4}\hslash^2\omega^2
\end{equation}
and
\begin{equation}\label{s5e11}
\sqrt{D - E^2} = 0.
\end{equation}
Thus, the energy $E$ computed in equation \eqref{s5e7} is the true ground
state energy of the harmomic oscillator and the corresponding trial function
of equation \eqref{s5e8} is the ground state eigenfunction.

\subsection{A particle in an infinite potential well}
The potential energy of the particle is
\begin{equation}\label{s5e12}
V(x) = \begin{cases}
0 & |x| < a \\
\infty & |x| \ge a.
\end{cases}
\end{equation}
The Hamiltonian in the well is
\begin{equation}\label{s5e13}
\hat{H} = -\frac{\hslash^2}{2m}\frac{d^2}{dx^2}.
\end{equation}
Although this is a trivial problem, let us pretend that we do not know
its solution and begin with a trial function
\begin{equation}\label{s5e14}
\phi(x;\alpha) = \begin{cases}
N(a^2 - x^2)(1 + \alpha x^2) & |x| < a \\
0 & |x| \ge a,
\end{cases}
\end{equation}
where $N$ is the normalization constant. We observe that $\phi$ is indeed a 
permissible wavefunction. The normalization constant is
\begin{equation}\label{s5e15}
N = \sqrt{\frac{315}{16 a^5(a^4\alpha^2 + 6a^2\alpha + 21)}}
\end{equation}
and
\begin{equation}\label{s5e16}
E(\alpha) = \int_{-\infty}^\infty \phi^\ast\hat{H}\phi dx = 
N^2 \left(-\frac{\hslash^2}{2m}\right)\left(-\frac{8a^3}{105}\right)
(11a^4\alpha^2 + 14a^2\alpha + 35)
\end{equation}
or
\begin{equation}\label{s5e17}
E(\alpha) = \frac{3\hslash^2}{4ma^2}
\left(\frac{11a^4\alpha^2+14a^2\alpha+35}{a^4\alpha^2+6a^2\alpha+21}\right).
\end{equation}
In order to get the extremum of $E$, we find it first derivative.
\begin{equation}\label{s5e18}
\frac{dE}{d\alpha} = \frac{3\hslash^2}{m}
\frac{13a^4\alpha^2 + 98 a^2\alpha + 21}{(a^4\alpha^2+6a^2\alpha+21)^2}.
\end{equation}
The condition for an extremum is
\begin{equation}\label{s5e19}
13a^2\alpha^2 + 98 a^2\alpha + 21 = 0
\end{equation}
or
\begin{equation}\label{s5e20}
\alpha = \frac{-98 \pm \sqrt{8512}}{26a^2}.
\end{equation}
We find it more convenient to express the roots as
\begin{eqnarray}
a^2\alpha_1 &=& -0.2207 \label{s5e21} \\
a^2\alpha_2 &=& -7.3177 \label{s5e22}
\end{eqnarray}
The normalization constants corresponding to these roots are
\begin{eqnarray}
N_1 &=& 0.9906 a^{-5/2} \label{s5e23} \\
N_2 &=& 0.8016 a^{-5/2} \label{s5e24}
\end{eqnarray}
and the values of $E$ are
\begin{eqnarray}
E(\alpha_1) &=& 1.2337187\frac{\hslash^2}{ma^2} \label{s5e25} \\
E(\alpha_2) &=& 12.7662813\frac{\hslash^2}{ma^2} \label{s5e26}
\end{eqnarray}
The quantity $D$ is
\begin{equation}\label{s5e27}
D = N^2\frac{\hslash^4}{m^2}\int_{-a}^a(6\alpha x^2 + 1 - a^2\alpha)^2dx
= N^2\frac{\hslash^4a}{m^2}\left(\frac{42a^4\alpha^2}{5} + 4a^2\alpha + 2
\right).
\end{equation}
The two values of $D$ corresponding to the two roots are
\begin{eqnarray}
D(\alpha_1) &=& 1.5234874\frac{\hslash^2}{m^2a^4} \label{s5e28} \\
D(\alpha_2) &=& 271.4765067\frac{\hslash^2}{m^2a^4} \label{s5e29}.
\end{eqnarray}
We can now compute the energy bounds corresponding to the two roots as
\begin{eqnarray}
\Delta(\alpha_1) &=& 0.0378488\frac{\hslash^2}{ma^2} \label{s5e30} \\
\Delta(\alpha_2) &=& 10.41626456\frac{\hslash^2}{ma^2} \label{s5e31}
\end{eqnarray}
Although we have computed $E$, $D$ and $\Delta$ for both roots, we know that
the ground state corresponds to $\alpha_1$. From equations \eqref{s5e25}
and \eqref{s5e30} we conclude that the true ground state energy $E_0$
is bounded by
\begin{equation}\label{s5e32}
1.1958699\frac{\hslash^2}{ma^2} \le E_0 \le 1.2715675\frac{\hslash^2}{ma^2}.
\end{equation}
We know that the true ground state energy is
\begin{equation}\label{s5e33}
E_0 = \frac{\hslash^2}{8ma^2}\pi^2 = 1.2337005\frac{\hslash^2}{ma^2}.
\end{equation}
In percentage terms, we have found the ground state energy to within $3.07
\%$ of the true value.

\subsection{The Hydrogen atom}
The Hamiltonian is
\begin{equation}\label{s5e34}
\hat{H} = -\frac{\hslash^2}{2m}\nabla^2 - \frac{e^2}{r}.
\end{equation}
Let us assume a spherically symmetric trial function
\begin{equation}\label{s5e35}
\phi(r; \alpha) = \begin{cases}
N\left(1 - \frac{r}{\alpha}\right) & r \le a \\
0 & r > a
\end{cases}
\end{equation}
where $N$ is the normalization constant. We first find the normalization 
constant using the relation
\[
N^2\int_0^\infty \left(1 - \frac{r}{\alpha}\right)^24\pi r^2dr = 1.
\]
Using the definition of $\phi$,
\[
N^2\int_0^\alpha\left(1 - \frac{r}{\alpha}\right)^2r^2dr = \frac{1}{4\pi}
\]
or
\[
N^2\frac{\alpha^3}{30} = \frac{1}{4\pi}
\]
from which we infer that
\begin{equation}\label{s5e36}
N = \sqrt{\frac{15}{2\pi}}\frac{1}{\alpha^{3/2}}.
\end{equation}
The trial function is thus,
\begin{equation}\label{s5e37}
\phi(r;\alpha) = \frac{1}{\alpha^{3/2}}\sqrt{\frac{15}{2\pi}}
\left(1-\frac{r}{\alpha}\right).
\end{equation}
We will now find $\hat{H}\phi$. To do that we write the full Laplacian.
\[
\nabla^2\phi = \frac{1}{r^2}\frac{\partial}{\partial r}\left(r^2\frac{
\partial\phi}{\partial r}\right) + \frac{1}{r^2\sin\theta}\frac{\partial}
{\partial\theta}\left(\sin\theta\frac{\partial\phi}{\partial\theta}
\right)+\frac{1}{r^2\sin^2\theta}\frac{\partial^2\phi}{\partial\varphi^2}.
\]
Since $\phi$ is a function of $r$ alone,
\[
\nabla^2\phi = \frac{1}{r^2}\frac{d}{dr}\left(r^2\frac{d\phi}{dr}\right)
= -\frac{2N}{\alpha r}
\]
so that
\begin{equation}\label{s5e38}
-\frac{\hslash^2}{2m}\int_{-\infty}^\infty\phi^\ast\nabla^2\phi 4\pi r^2dr
= \frac{5\hslash^2}{m\alpha^2}.
\end{equation}
Similarly, the expected value of the potential energy $\langle V \rangle$ is
\[
\int_{-\infty}^\infty\phi^\ast\left(-\frac{e^2}{r}\right)\phi 4\pi r^2dr = 
-4\pi e^2\int_{-\infty}^\infty r|\phi|^2dr = -4\pi e^2N^2\int_0^\alpha
\left(1 - \frac{r}{\alpha}\right)^2 r dr.
\]
Therefore,
\begin{equation}\label{s5e39}
\langle V \rangle = -\frac{30e^2}{\alpha}\int_0^\alpha
\left(1-\frac{r}{\alpha}\right)^2 rdr = -\frac{30 e^2}{\alpha}
\frac{\alpha^2}{12} = -\frac{15 e^2}{6\alpha}.
\end{equation}
From equations \eqref{s5e38} and \eqref{s5e39} we infer that
\begin{equation}\label{s5e40}
E(\alpha) = \frac{5\hslash^2}{m\alpha^2} - \frac{5e^2}{2\alpha}.
\end{equation}
The extremum of this function is at
\begin{equation}\label{s5e41}
\alpha_0 = \frac{4\hslash^2}{me^2}.
\end{equation}
The trial function is then
\begin{equation}\label{s5e42}
\phi(r) = \sqrt{\frac{15}{2\pi}}\frac{1}{\alpha_0^{3/2}}\left(1 - 
\frac{r}{\alpha_0}\right)
\end{equation}
and
\begin{equation}\label{s5e43}
E(\alpha_0) = -\frac{5}{16}\frac{me^4}{\hslash^4}.
\end{equation}

We will now find $D(\alpha_0)$. It is
\begin{equation}\label{s5e44}
\int_{-\infty}^\infty |\hat{H}\phi|^2dV = 
\frac{30}{\alpha_0^3}\int_0^\alpha\left(1 - \frac{r}{\alpha_0}\right)^2
r^2dr = \frac{30}{16}.\frac{106}{96}.\frac{m^2e^8}{\hslash^2}.
\end{equation}

The bounds to the variational calculation need
\[
\Delta^2 = D(\alpha_0) - E^2(\alpha_0) = \frac{505}{256}\frac{m^2e^8}
{\hslash^2}.
\]
The true ground state energy $E_0$ is thus
\begin{equation}\label{s5e45}
-\frac{5 + \sqrt{505}}{16}\frac{me^4}{\hslash^2} \le E_0 \le
\frac{-5 + \sqrt{505}}{16}\frac{me^4}{\hslash^2} 
\end{equation}
or
\begin{equation}\label{s5e46}
-1.7170\frac{me^4}{\hslash^2} \le E_0 \le 1.092\frac{me^4}{\hslash^2}.
\end{equation}
The true energy is
\begin{equation}\label{s5e47}
E_0 = -\frac{1}{8}\frac{m\hslash^2}{e^4}.
\end{equation}

\subsection{Helium like atoms}
The Hamiltonian in this cases is
\begin{equation}\label{s5e48}
\hat{H} = -\frac{\hslash^2}{2m}\left(\nabla_1^2 + \nabla_2^2\right) -
\frac{Ze^2}{r_1} - \frac{Ze^2}{r_2} + \frac{e^2}{r_{12}},
\end{equation}
where $m$ is the mass of the electron, $r_1$ is the distance of the first
electron from the nucleus, $r_2$ that of the second electron from the 
nucleus and $r_{12}$ is the distance between the two electrons. $Z$ is the
number of protons in the nucleus. $Z = 2$ for He atom, $3$ for Li$^+$ ion
and $4$ for Be$^{++}$ ion. We have partially solved this problem in section
\ref{s3}. We will sketch that part of the solution here before proceeding
to find the bounds of the variational estimate.

The trial wave function is
\begin{equation}\label{s5e49}
\phi(\vec{x}_1, \vec{x}_2) = \frac{\alpha^3}{\pi a_0^3}
\exp\left(-\frac{\alpha}{a_0}(r_1 + r_2)\right),
\end{equation}
where $a_0$ is the radius of the first Bohr orbit of the Hydrogen atom
and $\alpha$ is the variational parameter. We had used $Z^\prime$ in place
of $\alpha$ in section \ref{s3}. We prefer to use $\alpha$ here for sake
of consistency with the other examples. The trial function is a product of
\begin{eqnarray}
\phi_1(\vec{x}_1) &=& \left(\frac{\alpha^{3}}{\pi a_0^3}\right)^{1/2}
\exp\left(-\frac{\alpha r_1}{a_0}\right) \label{s5e50} \\
\phi_2(\vec{x}_2) &=& \left(\frac{\alpha^{3}}{\pi a_0^3}\right)^{1/2}
\exp\left(-\frac{\alpha r_2}{a_0}\right) \label{s5e51}
\end{eqnarray}
$\phi_1$ and $\phi_2$ are eigenfunctions of the hydrogenic Hamiltonians 
with atomic number $\alpha e$. That is,
\begin{eqnarray}
-\frac{\hslash^2}{2m}\nabla_1^2\phi_1 - \frac{\alpha e^2}{r_1} &=& 
-\alpha^2W_H\phi_1 \label{s5e52} \\
-\frac{\hslash^2}{2m}\nabla_2^2\phi_2 - \frac{\alpha e^2}{r_2} &=& 
-\alpha^2W_H\phi_2 \label{s5e53}, 
\end{eqnarray}
where 
\begin{equation}\label{s5e54}
W_H = \frac{e^2}{2a_0}.
\end{equation}
Following the steps outlined in section \ref{s3} we get
\begin{equation}\label{s5e55}
E(\alpha) = 
\left(-2\alpha^2 + 4\alpha(\alpha - Z) + \frac{5}{4}\alpha\right)W_H
\end{equation}
The extremum of this function occurs are
\begin{equation}\label{s5e56}
\alpha_0 = Z - \frac{5}{16}
\end{equation}
and
\begin{equation}\label{s5e57}
E(\alpha_0) = -2\left(Z - \frac{5}{16}\right)^2 W_H.
\end{equation}
In order to get bounds on the energy estimate, we need to compute
$D(\alpha)$ defined by equation \eqref{s4e2}
\[
D(\alpha) = \int_{-\infty}^\infty \int_{-\infty}^\infty 
|\hat{H}\phi|^2 dV_1 dV_2,
\]
where $dV_1$ and $dV_2$ are the volume elements in $\vec{x}_1$ and 
$\vec{x}_2$ coordinates. From equation \eqref{s3e11},
\[
\hat{H}\phi = -2\alpha^2W_H\phi + (Z - \alpha)e^2\left(\frac{1}{r_1}
+ \frac{1}{r_2}\right)\phi + \frac{e^2}{r_{12}}\phi.
\]
Define the constants
\begin{eqnarray}
A &=& -2\alpha^2 W_H \label{s5e58} \\
B &=& (Z - \alpha) e^2 \label{s5e59} \\
C &=& e^2 \label{s5e60}
\end{eqnarray}
so that
\begin{equation}\label{s5e61}
\hat{H}\phi = A\phi + B\left(\frac{1}{r_1} + \frac{1}{r_2}\right)\phi +
\frac{C}{r_{12}}\phi
\end{equation}
and
\begin{eqnarray*}
|\hat{H}\phi|^2 &=& A^2|\phi|^2 + B^2\left(\frac{1}{r_1^2} + 
  \frac{1}{r_2^2} + \frac{2}{r_1r_2}\right)|\phi|^2 + \\
  & & \frac{C^2}{r_{12}^2}|\phi|^2 + 2AB\left(\frac{1}{r_1}+\frac{1}{r_2}
  \right)|\phi|^2 + 2\frac{AC}{r_{12}}|\phi|^2 + \\
  & & 2BC\left(\frac{1}{r_1r_{12}} + \frac{1}{r_2r_{12}}\right)|\phi|^2.
\end{eqnarray*}
Thus,
\begin{equation}\label{s5e62}
D(\alpha) = A^2I_1 + B^2(I_2 + I_3 + 2I_4) + C^2I_5 + 2AB(I_6 + I_7) 
    + 2ACI_8 + 2BC(I_9 + I_{10}),
\end{equation}
where the ten integrals are
\begin{eqnarray}
I_1 &=& \int_{-\infty}^\infty\int_{-\infty}^\infty |\phi|^2 dV_1 dV_2 
        \label{s5e63} \\
I_2 &=& \int_{-\infty}^\infty\int_{-\infty}^\infty\frac{|\phi|^2}{r_1^2}
        dV_1dV_2\label{s5e64}\\
I_3 &=& \int_{-\infty}^\infty\int_{-\infty}^\infty\frac{|\phi|^2}{r_2^2}
        dV_1dV_2\label{s5e65}\\
I_4 &=& \int_{-\infty}^\infty\int_{-\infty}^\infty\frac{|\phi|^2}{r_1r_2}
        dV_1dV_2\label{s5e66}\\
I_5 &=& \int_{-\infty}^\infty\int_{-\infty}^\infty\frac{|\phi|^2}{r_{12}^2}
        dV_1dV_2\label{s5e67}\\
I_6 &=& \int_{-\infty}^\infty\int_{-\infty}^\infty\frac{|\phi|^2}{r_1}
        dV_1dV_2\label{s5e68}\\
I_7 &=& \int_{-\infty}^\infty\int_{-\infty}^\infty\frac{|\phi|^2}{r_2}
        dV_1dV_2\label{s5e69}\\
I_8 &=& \int_{-\infty}^\infty\int_{-\infty}^\infty\frac{|\phi|^2}{r_{12}}
        dV_1dV_2\label{s5e70}\\
I_9 &=& \int_{-\infty}^\infty\int_{-\infty}^\infty\frac{|\phi|^2}
        {r_1r_{12}}dV_1dV_2 \label{s5e71}\\
I_{10} &=& \int_{-\infty}^\infty\int_{-\infty}^\infty\frac{|\phi|^2}
        {r_2r_{12}}dV_1dV_2 \label{s5e72}
\end{eqnarray}
We will now compute each of these integrals. We first note that
\[
|\phi|^2 = 
\frac{\alpha^6}{\pi^2 a_0^6}\exp\left(-\frac{2\alpha}{a_0}(r_1+r_2)\right)
= \frac{\alpha^6}{\pi^2 a_0^6}\exp\left(-\frac{2\alpha}{a_0}r_1\right)
\exp\left(-\frac{2\alpha}{a_0}r_2\right)  
\]
Thus,
\begin{eqnarray*}
I_1 &=& \frac{\alpha^6}{\pi^2 a_0^6}
\int_0^\infty\int_0^{2\pi}\int_0^\pi\exp\left(-\frac{2\alpha}{a_0}r_1
	\right) r_1^2dr_1\sin\theta_1d\theta_1d\varphi_1 \\
 & & 
\int_0^\infty\int_0^{2\pi}\int_0^\pi\exp\left(-\frac{2\alpha}{a_0}r_2
	\right) r_2^2dr_2\sin\theta_2d\theta_2d\varphi_2 \\
 &=& \frac{\alpha^6}{\pi^2 a_0^6}\cdot 16\pi^2 \cdot
 \int_0^\infty\exp\left(-\frac{2\alpha}{a_0}r_1\right)r_1^2dr_1 \\
 & & \int_0^\infty\exp\left(-\frac{2\alpha}{a_0}r_2\right)r_2^2dr_2 
\end{eqnarray*}
Using equation \eqref{a1e1},
\[
\int_0^\infty\exp\left(-\frac{2\alpha}{a_0}r_1\right)r_1^2dr_1 = 
\frac{2!}{(2\alpha/a_0)^3} = \frac{2a_0^3}{8\alpha^3}
\]
so that
\[
I_1 = \frac{\alpha^6}{\pi^2 a_0^6}\cdot 16\pi^2 \cdot
 \frac{2a_0^3}{8\alpha^3}\frac{2a_0^3}{8\alpha^3}
\]
or
\begin{equation}\label{s5e73}
I_1 = 1.
\end{equation}
We now go to $I_2$.
\begin{eqnarray*}
I_2 &=& \frac{\alpha^6}{\pi^2 a_0^6}
\int_0^\infty\int_0^{2\pi}\int_0^\pi
\frac{1}{r_1^2}\exp\left(-\frac{2\alpha}{a_0}r_1\right)
r_1^2dr_1\sin\theta_1d\theta_1d\varphi_1 \\
 & & 
\int_0^\infty\int_0^{2\pi}\int_0^\pi\exp\left(-\frac{2\alpha}{a_0}r_2
	\right) r_2^2dr_2\sin\theta_2d\theta_2d\varphi_2 \\
 &=& \frac{\alpha^6}{\pi^2 a_0^6}\cdot 16\pi^2 \cdot
 \int_0^\infty\exp\left(-\frac{2\alpha}{a_0}r_1\right)dr_1 
 \int_0^\infty\exp\left(-\frac{2\alpha}{a_0}r_2\right)r_2^2dr_2 
\end{eqnarray*}
Using equation \eqref{a1e1} and \eqref{a1e2}
\[
I_2 = \frac{\alpha^6}{\pi^2 a_0^6}\cdot 16\pi^2 \cdot \frac{a_0}{2\alpha}
\cdot \frac{2a_0^3}{8\alpha^3}.
\]
Thus,
\begin{equation}\label{s5e74}
I_2 = \frac{2\alpha^2}{a_0^2}.
\end{equation}
$I_3$ is similar to $I_2$ except for an interchange of $r_1$ and $r_2$ so
that
\begin{equation}\label{s5e75}
I_3 = \frac{2\alpha^2}{a_0^2}.
\end{equation}
Moving to $I_4$,
\begin{eqnarray*}
I_4 &=& \frac{\alpha^6}{\pi^2 a_0^6}
\int_0^\infty\int_0^{2\pi}\int_0^\pi
\frac{1}{r_1}\exp\left(-\frac{2\alpha}{a_0}r_1\right)
r_1^2dr_1\sin\theta_1d\theta_1d\varphi_1 \\
 & & 
\int_0^\infty\int_0^{2\pi}\int_0^\pi
\frac{1}{r_2}\exp\left(-\frac{2\alpha}{a_0}r_2\right)
r_2^2dr_2\sin\theta_2d\theta_2d\varphi_2 \\
&=& \frac{\alpha^6}{\pi^2 a_0^6} \cdot 16\pi^2 \cdot
\int_0^\infty r_1\exp\left(-\frac{2\alpha}{a_0}r_1\right)dr_1
\int_0^\infty r_2\exp\left(-\frac{2\alpha}{a_0}r_2\right)dr_2 \\
&=& \frac{\alpha^6}{\pi^2 a_0^6} \cdot 16\pi^2 \cdot
\frac{a_0^2}{4\alpha^2}\frac{a_0^2}{4\alpha^2}
\end{eqnarray*}
Thus,
\begin{equation}\label{s5e76}
I_4 = \frac{\alpha^2}{a_0^2}.
\end{equation}
Consider
\[
I_5 = \int_{-\infty}^\infty\int_{-\infty}^\infty \frac{|\phi|^2}{r_{12}^2}
dV_1 dV_2.
\]
From equations \eqref{s5e50} and \eqref{s5e51},
\[
I_5 = \frac{\alpha^6}{\pi^2 a_0^6}
\int_{-\infty}^\infty\int_{-\infty}^\infty
\frac{r_1^2e^{-2\alpha r_1/a_0}r_2^2e^{-2\alpha r_2/a_0}}{r_{12}^2}
dV_1 dV_2.
\]
Without the constant multiplying it, the integral is the energy of two
spherically symmetric charge distributions in CGS units. Let us first
calculate the potential due to one of them, say 
$r_1^2\exp(-2\alpha r_1/a_0)$. The potential due to a spherical shell of 
radius $r_1$ and a total charge $r_1^2\exp(-2\alpha r_1/a_0)\times 
(4\pi r_1^2)$ is
\begin{equation}\label{s5e77}
dV(r) = \begin{cases}
4\pi r_1^4\exp(-2\alpha r_1/a_0) r_1^{-1} & \;\text{if}\; r < r_1 \\
4\pi r_1^4\exp(-2\alpha r_1/a_0) r^{-1} & \;\text{if}\; r > r_1 \\
\end{cases}
\end{equation}
so that
\[
V(r) = \frac{4\pi}{r}\int_0^r r_1^4\exp\left(-\frac{2\alpha r_1}{a_0}
       \right)dr_1 + 4\pi\int_r^\infty r_1^3\exp\left(-\frac{\alpha r_1}
       {a_0}\right)dr_1.
\]
Let us simplify this expression by putting $K = 2\alpha/a_0$. Thus,
\[
V(r) = \frac{4\pi}{r}\int_0^r r_1^4e^{-Kr_1}dr_1 + 
       4\pi\int_r^\infty r_1^3 e^{-Kr_1}dr_1.
\]
Using equations \eqref{a1e6} and \eqref{a1e10} we get
\begin{equation}\label{s5e78}
V(r) = -\frac{4\pi e^{-Kr}}{K^2}r^2 - \frac{24\pi e^{-Kr}}{K^3}r - 
       \frac{72\pi e^{-Kr}}{K^4} + \frac{96\pi}{K^5}\frac{1 - e^{-Kr}}{r}
\end{equation}
Integral $I_5$ then is
\[
I_5 = \frac{\alpha^6}{\pi^2 a_0^6}\int_{-\infty}^\infty V(r_2)r_2^2
\exp\left(-\frac{2\alpha}{a_0}r_2\right)dV_2.
\]
Integrating over the angles and writting in terms of $K$ it is
\[
I_5 = \frac{\alpha^6}{\pi^2 a_0^6}\cdot 4\pi\cdot 
      \int_0^\infty V(r_2)r_2^4 e^{-Kr_2} dr_2
\]
Substituting \eqref{s5e78} on the right hand side, we get
\begin{eqnarray*}
I_5 &=& -\frac{16\alpha^6}{a_0^6K^2}\int_0^\infty r_2^6e^{-2Kr_2}dr_2 - 
  \frac{96\alpha^6}{a_0^6K^3}\int_0^\infty r_2^5e^{-2Kr_2}dr_2 - \\
 & & \frac{288\alpha^6}{a_0^6K^4}\int_0^\infty r_2^4e^{-2Kr_2}dr_2 + 
 \frac{384\alpha^6}{a_0^6 K^5}\int_0^\infty(e^{-Kr_2}-e^{-2Kr_2})r_2^3dr_2
\end{eqnarray*}
All the integrals are a form of equation \eqref{a1e1}. Thus,
\begin{eqnarray*}
I_5 &=& -\frac{16\alpha^6}{a_0^6K^2}\frac{6!}{(2K)^7} -
 \frac{96\alpha^6}{a_0^6K^3}\frac{5!}{(2K)^6} - 
 \frac{288\alpha^6}{a_0^6K^4}\frac{4!}{(2K)^5} + \\
 & & \frac{384\alpha^6}{a_0^6 K^5}\left(\frac{3!}{K^4} - \frac{3!}{(2K)^4}
	 \right)
\end{eqnarray*}
or
\[
I_5 = -\frac{90\alpha^6}{a_0^6K^9} - \frac{180\alpha^6}{a_0^6K^9} - 
 \frac{216\alpha^6}{a_0^6K^9} + \frac{372\alpha^6}{a_0^6 K^9}
 = -\frac{114\alpha^6}{a_0^6K^9}
\]
Substituting for $K$ we get
\begin{equation}\label{s5e79}
I_5 = \frac{57}{256}\frac{a_0^3}{\alpha^3}.
\end{equation}

Moving to $I_6$,
\begin{eqnarray*}
I_6 &=& \frac{\alpha^6}{\pi^2 a_0^6}
\int_0^\infty\int_0^{2\pi}\int_0^\pi
\frac{1}{r_1}\exp\left(-\frac{2\alpha}{a_0}r_1\right)
r_1^2dr_1\sin\theta_1d\theta_1d\varphi_1 \\
 & & 
\int_0^\infty\int_0^{2\pi}\int_0^\pi
\exp\left(-\frac{2\alpha}{a_0}r_2\right)
r_2^2dr_2\sin\theta_2d\theta_2d\varphi_2 \\
&=& \frac{\alpha^6}{\pi^2 a_0^6} \cdot 16\pi^2 \cdot
\int_0^\infty r_1\exp\left(-\frac{2\alpha}{a_0}r_1\right)dr_1
\int_0^\infty r_2^2\exp\left(-\frac{2\alpha}{a_0}r_2\right)dr_2 \\
&=& \frac{\alpha^6}{\pi^2 a_0^6} \cdot 16\pi^2 \cdot
\frac{a_0^2}{4\alpha^2}\frac{a_0^3}{8\alpha^3}\cdot 2!
\end{eqnarray*}
Thus,
\begin{equation}\label{s5e80}
I_6 = \frac{\alpha}{a_0}.
\end{equation}
$I_7$ is same as $I_6$ except for an interchange of $r_1$ and $r_2$. Thus,
\begin{equation}\label{s5e81}
I_7 = \frac{\alpha}{a_0}.
\end{equation}
Consider
\[
I_8 = \int_{-\infty}^\infty\int_{-\infty}^\infty\frac{|\phi|^2}{r_{12}}
dV_1 dV_2.
\]
We must evaluate it similar to the way we evaluated $I_5$. Using the same
terminology as we used there, we first compute the potential due to the
spherically symmetric charge distribution $\exp(-2\alpha r_1/a_0)$. The
potential due to a spherical shell with total charge $4\pi r_1^2 \exp(
-2\alpha r_1/a_0)dr_1$ is
\[
dV(r) = \begin{cases}
4\pi r_1^2e^{-Kr_1}dr_1 \cdot r_1^{-1} & \;\text{if}\; r < r_1 \\
4\pi r_1^2e^{-Kr_1}dr_1 \cdot r^{-1}   & \;\text{if}\; r > r_1
\end{cases}
\]
so that
\[
V(r) = \frac{4\pi}{r}\int_0^r r_1^2e^{-Kr_1}dr_1 + 
       4\pi\int_r^\infty r_1 e^{-r_1}dr_1
\]
or
\[
V(r) = \frac{4\pi}{r}\left(\frac{2}{K^3} - \frac{e^{-Kr}}{K}\left(r^2 +
\frac{2r}{K} + \frac{2}{K^2}\right)\right) + 4\pi e^{-Kr}\left(\frac{r}
{K} + \frac{1}{K}\right).
\]
Simplifying the expression, we get
\[
V(r) = \frac{8\pi}{K^3}\frac{1 - e^{-Kr}}{r} + \frac{4\pi e^{-Kr}}{K}
\left(1 - \frac{2}{K}\right)
\]
so that
\[
I_8 = \frac{\alpha^6}{\pi^2 a_0^6}\int_{-\infty}^\infty V(r_2)e^{-Kr_2}dr_2
\]
Evaluating the angular integrals,
\[
I_8 = \frac{\alpha^6}{\pi^2 a_0^6}\cdot 4\pi \int_0^\infty r^2
V(r_2)e^{-Kr_2}dr_2
\]

After substituting for $V$,
\[
I_8 = \frac{\alpha^6}{\pi^2 a_0^6} \cdot 16\pi^2 \left\{
\frac{2}{K^3}\int_0^\infty r_2(e^{-Kr_2} - e^{-2Kr_2})dr_2 + 
\frac{K-2}{K^2}\int_0^\infty r_2^2 e^{-2Kr_2}dr_2\right\}
\]
From equation \eqref{a1e1},
\[
I_8 = \frac{16\alpha^6}{a_0^6}\left\{\frac{2}{K^3}\left(\frac{1}{K^2} -
    \frac{1}{4K^2}\right) + \frac{K-2}{K^2}\frac{2}{4K^2}\right\}
 = \frac{16\alpha^6}{a_0^6K^4}\left(\frac{1}{2K} + \frac{K-2}{2}\right).
\]
Putting $K = 2\alpha/a_0$ gives
\begin{equation}\label{s5e82}
I_8 = \frac{\alpha^2}{a_0^2}\left(\frac{1}{4}\frac{a_0}{\alpha} + 
      \frac{\alpha}{a_0} - 1\right).
\end{equation}
We next evaluate
\[
I_9 = \int_{-\infty}^\infty\int_{\infty}^\infty \frac{|\phi|^2}{r_1r_{12}}
dV_1dV_2
\]
The presence of $r_{12}$ suggests that we should evaluate this integral 
the way we evaluated $I_5$ and $I_8$. The potential due to a spherical
shell with total charge $4\pi r_1^2\exp(-2\alpha r_1/a_0)/r_1$ is
\[
dV(r) = \begin{cases}
4\pi e^{-Kr_1}dr_1 & \;\text{if}\; r < r_1 \\
4\pi r_1 e^{-Kr_1}\cdot r^{-1} & \;\text{if}\; r > r_1
\end{cases}
\]
Thus,
\[
V(r) = \frac{4\pi}{r}\int_0^r r_1e^{-Kr_1}dr_1 + 
4\pi\int_r^\infty e^{-Kr_1}dr_1 = \frac{4\pi}{K^2}\frac{1 - e^{-Kr}}{r}.
\]
Thus,
\[
I_9 = \frac{\alpha^6}{\pi^2 a_0^6}
\int_{-\infty}^\infty V(r_2)e^{-Kr_2}dr_2 = \frac{64\alpha^6}{a_0^6K^2}
\int_0^\infty r_2(e^{-Kr_2} - e^{-2Kr_2})dr_2.
\]
Upon evaluating the integrals, we get
\[
I_9 = \frac{64\alpha^6}{a_0^6K^2}\cdot\frac{3}{4K^4}.
\]
Putting the value of $K$ we get
\begin{equation}\label{s5e83}
I_9 = \frac{3\alpha^2}{a_0^2}.
\end{equation}
The integral $I_{10}$ is same as $I_9$ except for an interchange of
variables. Thus,
\begin{equation}\label{s5e84}
I_{10} = \frac{3\alpha^2}{a_0^2}.
\end{equation}

We have now evaluated all the integrals needed to get an expression of
$D(\alpha)$.
\begin{eqnarray}
I_1 &=& \int_{-\infty}^\infty\int_{-\infty}^\infty |\phi|^2 dV_1 dV_2 
 = 1 \label{s5e85} \\
I_2 &=& \int_{-\infty}^\infty\int_{-\infty}^\infty\frac{|\phi|^2}{r_1^2}
        dV_1dV_2 = \frac{2\alpha^2}{a_0^2} \label{s5e86}\\
I_3 &=& \int_{-\infty}^\infty\int_{-\infty}^\infty\frac{|\phi|^2}{r_2^2}
        dV_1dV_2 = \frac{2\alpha^2}{a_0^2} \label{s5e87}\\
I_4 &=& \int_{-\infty}^\infty\int_{-\infty}^\infty\frac{|\phi|^2}{r_1r_2}
        dV_1dV_2 = \frac{\alpha^2}{a_0^2} \label{s5e88}\\
I_5 &=& \int_{-\infty}^\infty\int_{-\infty}^\infty\frac{|\phi|^2}{r_{12}^2}
        dV_1dV_2 = \frac{57}{256}\frac{a_0^3}{\alpha^3} \label{s5e89}\\
I_6 &=& \int_{-\infty}^\infty\int_{-\infty}^\infty\frac{|\phi|^2}{r_1}
        dV_1dV_2 = \frac{\alpha}{a_0} \label{s5e90} \\
I_7 &=& \int_{-\infty}^\infty\int_{-\infty}^\infty\frac{|\phi|^2}{r_2}
        dV_1dV_2 = \frac{\alpha}{a_0} \label{s5e91} \\
I_8 &=& \int_{-\infty}^\infty\int_{-\infty}^\infty\frac{|\phi|^2}{r_{12}}
        dV_1dV_2 = \frac{\alpha^2}{a_0^2}\left(\frac{a_0}{4\alpha} + 
	\frac{\alpha}{a_0}- 1 \right) \label{s5e92}\\
I_9 &=& \int_{-\infty}^\infty\int_{-\infty}^\infty\frac{|\phi|^2}
        {r_1r_{12}}dV_1dV_2 = \frac{3\alpha^2}{a_0^2} \label{s5e93}\\
I_{10} &=& \int_{-\infty}^\infty\int_{-\infty}^\infty\frac{|\phi|^2}
        {r_2r_{12}}dV_1dV_2 = \frac{3\alpha^2}{a_0^2} \label{s5e94}
\end{eqnarray}
Recall that the extremum of $E$ occurs at $\alpha_0 = Z - 5/16$. We now
get an expression for $D(\alpha_0)$. To do so, we use the equations 
\eqref{s5e58} to \eqref{s5e60} to get expressions for $A, B, C$, 
equation \eqref{s5e62} for $D(\alpha)$ and equations \eqref{s5e85} to
\eqref{s5e94} for the ten integrals to get
\begin{eqnarray*}
D(\alpha) &=& 4\alpha^4W_H^2 + \frac{6\alpha^2}{a_0^2}(Z - \alpha)^2e^4
 + \frac{57}{256}e^4\frac{a_0^3}{\alpha^3} - 4W_H\frac{e^2}{a_0}
 \alpha^3(Z - \alpha) \\
 & & -4W_He^2\frac{\alpha^4}{a_0^2}\left(\frac{a_0}{4\alpha} +
    \frac{\alpha}{a_0} - 1\right) + \frac{12e^4}{a_0^2}(Z - \alpha)
    \alpha^2.
\end{eqnarray*}
From equation \eqref{s5e56}
\[
\alpha_0 = Z - \frac{5}{16},
\]
so that
\begin{eqnarray}
D(\alpha_0) &=& 4W_H^2\left(Z - \frac{5}{16}\right)^4 - \frac{125}{128}
\frac{e^4}{a_0^2}\left(Z - \frac{5}{16}\right)^2 + \frac{57}{256}e^4a_0^3
\left(Z - \frac{5}{16}\right)^{-3} - \nonumber \\
 & & \frac{5}{4}W_H\frac{e^2}{a_0}\left(Z - \frac{5}{16}\right)^3 -
 W_H\frac{e^2}{a_0^2}\left(Z - \frac{5}{16}\right)^3 - 4W_H\frac{e^2}
{a_0^3}\left(Z - \frac{5}{16}\right)^5 + \nonumber \\
 & & 4W_H\frac{e^2}{a_0^2}\left(Z - \frac{5}{16}\right)^4 + \frac{15}{4}
 \frac{e^4}{a_0^2}\left(Z - \frac{5}{16}\right)^2 \label{s5e95}
\end{eqnarray}
From equation \eqref{s5e57}, we get
\[
E^2(\alpha_0) = 4W_H^2\left(Z - \frac{5}{16}\right)^4,
\]
which is precisely the first term of equation \eqref{s5e95}. Therefore,
$\Delta^2(\alpha_0) = D(\alpha_0) - E^2(\alpha_0)$ evaluates to
\begin{eqnarray}
\Delta^2(\alpha_0) &=& - \frac{125}{128}
\frac{e^4}{a_0^2}\left(Z - \frac{5}{16}\right)^2 + \frac{57}{256}e^4a_0^3
\left(Z - \frac{5}{16}\right)^{-3} - \nonumber \\
 & & \frac{5}{4}W_H\frac{e^2}{a_0}\left(Z - \frac{5}{16}\right)^3 -
 W_H\frac{e^2}{a_0^2}\left(Z - \frac{5}{16}\right)^3 - 4W_H\frac{e^2}
{a_0^3}\left(Z - \frac{5}{16}\right)^5 + \nonumber \\
 & & 4W_H\frac{e^2}{a_0^2}\left(Z - \frac{5}{16}\right)^4 + \frac{15}{4}
 \frac{e^4}{a_0^2}\left(Z - \frac{5}{16}\right)^2 \label{s5e96}
\end{eqnarray}


\section{Some standard integrals}\label{a}
\begin{enumerate}
\item Let
\[
I_n = \int_0^\infty x^n e^{-Kx}dx.
\]
Integrating by parts gives the recurrence relation
\[
I_n = \frac{n}{K}I_{n-1}.
\]
We can solve it if we know 
\[
I_0 = \int_0^\infty e^{-Kx}dx = \frac{1}{K}.
\]
Thus
\begin{equation}\label{a1e1}
\int_0^\infty x^n e^{-Kx}dx = \frac{n!}{K^{n+1}}.
\end{equation}

\item Let
\[
I_n = \int_0^r x^n e^{-Kx}dx,
\]
where $n \ge 0$ is an integer.  We can integrate it by parts to get 
\[
I_n = -\frac{r^n e^{-Kr}}{K} + \frac{n}{K}\int_0^r e^{-Kx}x^{n-1}dx
\]
and write it as a recurrence relation
\[
I_n = -\frac{r^n e^{-Kr}}{K} + \frac{n}{K}I_{n-1}.
\]
We can solve the recurrence relation if we know the initial condition.
\[
I_0 = \int_0^r e^{-Kx}dx = \frac{1}{K} - \frac{e^{-Kr}}{K}.
\]
The solution of the recurrence relation is
\begin{equation}\label{a1e2}
\int_0^r x^n e^{-Kx}dx = \frac{n!}{K^{n+1}} - \frac{e^{-Kr}}{K}
\sum_{j=0}^n\frac{n!}{(n-j)!}\frac{r^{n-j}}{K^j}.
\end{equation}
We now get the following integrals
\begin{eqnarray}
\int_0^r xe^{-Kx}dx &=& \frac{1}{K^2} - 
         \frac{e^{-Kr}}{K}\left(r + \frac{1}{K}\right) \label{a1e3} \\
\int_0^r x^2e^{-Kx}dx  &=& \frac{2}{K^3} - \frac{e^{-Kr}}{K}\left(r^2 + 
	\frac{2r}{K} + \frac{2}{K^2}\right) \label{a1e4} \\
\int_0^r x^3e^{-Kx}dx  &=& \frac{6}{K^4} - \frac{e^{-Kr}}{K}\left(r^3 + 
	\frac{3r^2}{K}+\frac{6r}{K^2}+\frac{6}{K^3}\right) \label{a1e5} \\
\int_0^r x^4e^{-Kx}dx  &=& \frac{24}{K^5} - \frac{e^{-Kr}}{K}\left(r^4 + 
	\frac{4r^3}{K} + 
	\frac{12r^2}{K^2} + \frac{24r}{K^3} + \frac{24}{K^4}\right)
        \label{a1e6}
\end{eqnarray}
\item Let
\[
I_n = \int_a^\infty x^n e^{-Kx}dx.
\]
Integration by parts gives us
\[
I_n = \frac{e^{-Ka}}{K}a^n + \frac{n}{K}I_{n-1}.
\]
The initial value of this recurrence relation is
\[
I_0 = \frac{e^{-Ka}}{K}
\]
and its solution is
\begin{equation}\label{a1e7}
\int_a^\infty x^n e^{-Kx}dx = 
e^{-Ka}
\left(\sum_{j=1}^{n+1}\frac{n!}{(n+1-j)!}\frac{a^{n+1-j}}{K^j}\right).
\end{equation}
We now get the following integrals
\begin{eqnarray}
\int_a^\infty xe^{-Kx}dx &=& e^{-Ka}\left(\frac{a}{K} + \frac{1}{K}\right)
    \label{a1e8} \\
\int_a^\infty x^2e^{-Kx}dx &=& e^{-Ka}\left(\frac{a^2}{K} + \frac{2a}{K^2}
	+ \frac{2}{K^3}\right) \label{a1e9} \\
\int_a^\infty x^3e^{-Kx}dx &=& e^{-Ka}\left(\frac{a^3}{K} + \frac{3a^2}{K^2}
	+ \frac{6a}{K^3} + \frac{6}{K^4}\right) \label{a1e10} \\
\int_a^\infty x^4e^{-Kx}dx &=& e^{-Ka}\left(\frac{a^4}{K} + \frac{4a^3}{K^2}
	+ \frac{12a^2}{K^3} + \frac{24a}{K^4}
	+ \frac{24}{K^5} \right) \label{a1e11}
\end{eqnarray}
\end{enumerate}
\end{document}
