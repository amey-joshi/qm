\documentclass{article}
\usepackage{amsmath, amssymb, amsthm, amsfonts}
\numberwithin{equation}{section}
\begin{document}
\title{Variational Methods in Quantum Mechanics}
\author{Amey Joshi}
\maketitle
\section{Introduction}\label{s1}
\section{The variation method}\label{s2}
The variation method relies on the fact that the quantity
\begin{equation}\label{s2e1}
E = \int\phi^\ast\hat{H}\phi dV
\end{equation}
is greater than the ground state energy eigenvalue of the hamiltonian $\hat{H}$. 
The function $\phi$ is an arbitrary square integrable function which satisfies
the appropriate boundary conditions and therefore can be a candidate solution
of the Schr\"{o}dinger equation $\hat{H}\phi = E\phi$. We call $\phi$ the
variation function. As the set of eigenfunction of $\hat{H}$ form a complete
orthonormal set $\{\psi_0, \psi_1, \ldots\}$, we can write $\phi$ as
\begin{equation}\label{s2e2}
\phi = \sum_{n \ge 0}a_n\psi_n.
\end{equation}
Substituting equation \eqref{s2e2} into equation \eqref{s2e1}, we get
\[
E = \sum_{m,n \ge 0}a_m^\ast a_n \int \psi_m^\ast \hat{H}\psi_n dV.
\]
If $\hat{H}\psi_n = W_n\psi_n$ then,
\[
E = \sum_{m,n \ge 0}a_m^\ast a_n W_n \int\psi_m^\ast\psi_n dV = \sum_{n \ge 0}|a_n|^2W_n.
\]
If $W_0$ is the eigenvalue of the ground state then
\begin{equation}\label{s2e3}
E - W_0 = \sum_{n \ge 0}|a_n|^2(W_n - W_0),
\end{equation}
where we have used the fact that $\sum_{n \ge 0}|a_n|^2 = 1$. Since $W_n \ge W_0$
for all $n$, we readily infer that
\begin{equation}\label{s2e4}
E \ge W_0.
\end{equation}
Thus, the energy calculated using equation \eqref{s2e1} is an \emph{upper bound}
to the ground state eigenvalue of the hamiltonian $\hat{H}$. 

In order to use equation \eqref{s2e4} for finding approximate ground state
eigenfunctions of $\hat{H}$ one normally chooses $\phi$ to depend on some 
parameter, say $\beta$. Thus we choose, $\phi \equiv \phi(\beta, \vec{x})$
so that equation \eqref{s2e4} becomes
\begin{equation}\label{s2e5}
E(\beta) \ge W_0.
\end{equation}
We then choose a $\beta$, say $\beta_0$ that minimizes $E(\beta)$. The function
$\phi(\beta_0, \vec{x})$ is also the closest to the true ground state 
eigenfunction $\psi$, where the `closeness' is measured as the usual $L^2$ norm.

Now suppose that we know the eigenvalues of the ground state and the first 
excited state of the system. We can use them to find out how close the 
function $\phi$ that minimizes $E - W_0$ is to $\psi_0$, the true but unknown
ground state eigenfunction. We mentioned that a measure of closeness between
the functions is the $L^2$ norm of their differences. Let us therefore define
\begin{equation}\label{s2e6}
\epsilon = \int (\phi - \psi_0)^\ast(\phi - \psi)dV.
\end{equation}
If we assume that both $\phi$ and $\psi_0$ are normalised then
\[
\epsilon = 2 - \int\phi^\ast\psi_0 dV - \int\phi\psi_0^\ast dV
\]
Using equation \eqref{s2e2},
\[
\epsilon = 2 - \sum_{n \ge 0}a_n^\ast\int\psi_n^\ast\psi_0 dV - \sum_{n \ge 0}a_n\int\psi_n\psi_0^\ast dV.
\]
Using the orthonormality of the set $\{\psi_0, \psi_1, \ldots\}$, we get
\begin{equation}\label{s2e7}
\epsilon = 2(1 - \Re(a_0)),
\end{equation}
where $\Re(\cdot)$ denotes the real value of a complex number. 

We can write equation \eqref{s2e3} as
\[
E - W_0 = \sum_{n \ge 0}|a_n|^2(W_n - W_0) \ge (W_1 - W_0)\sum_{n \ge 1}|a_n|^2.
\]
Since $\sum_{n \ge 0}|a_n|^2 = 1$, we can as well write
\begin{equation}\label{s2e8}
E - W_0 \ge (W_1 - W_0)(1 - |a_0|^2).
\end{equation}

When $\epsilon \ll 1$, equation \eqref{s2e7} suggests that $\Re(a_0)$ is very
close to $1$. Therefore, $1 + \Re(a_0) \approx 1 + 1 = 2$ and therefore,
\begin{equation}\label{s2e9}
\epsilon \approx (1 - \Re^2(a_0)).
\end{equation}
Further, since $\sum_{n \ge 0}|a_n|^2 = 1$, if $\Re(a_0)$ is very close to $1$
then $\Im(a_0)$ must be very close to zero. Therefore, $|a_0|^2 \approx
\Re(a_0)^2$, allowing us to write \eqref{s2e8} as
\begin{equation}\label{s2e10}
E - W_0 \ge (W_1 - W_0)(1 - \Re^2(a_0)).
\end{equation}
Combining it with equation \eqref{s2e9} we get
\begin{equation}\label{s2e11}
\frac{E - W_0}{W_1 - W_0} \ge \epsilon.
\end{equation}
We can use \eqref{s2e7} to get 
\begin{equation}\label{s2e12}
1 - \Re(a_0) \le \frac{1}{2}\frac{E - W_0}{W_1 - W_0}.
\end{equation}
$W_0$ and $W_1$ are the experimentally determined quantities while $E$ is
calculated from our choice of $\phi$. Equation \eqref{s2e12} tells how far
does $\Re(a_0)$ deviate from $1$ given these quantities. The closer $\Re(a_0)$
is to $1$, the better is $\phi$ as an approximation to $\psi_0$.

\section{Examples of using variation method}\label{s3}
\subsection{Helium-like atoms}
Consider an atom like He or ions like Li$^{+}$ or Be$^{2+}$. They all have
two electrons and $Z$ protons, where $Z = 2, 3, 4$ respectively for He, Li$^{+}$,
Be$^{2+}$. The hamiltonian, in cgs units, is
\begin{equation}\label{s3e1}
\hat{H} = -\frac{\hslash^2}{2m}\left(\nabla_1^2 + \nabla_2^2\right)\psi(\vec{x}_1, \vec{x}_2)
- \frac{Ze^2}{r_1} - \frac{Ze^2}{r_2} + \frac{e^2}{r_{12}}.
\end{equation}
Here $\vec{x}_1$ and $\vec{x}_2$ are the position vectors of the two electrons
with respect to the nucleus at the origin, $r_1 = |\vec{x}_1|, r_2 = |\vec{x}_2|,
r_{12} = |\vec{x}_1 - \vec{x}_2|$, $e$ is the electronic charge, $m$ the 
electronic mass, $\nabla_1^2$ is the laplacian in terms of $\vec{x}_1$ coordinates 
and $\nabla_2^2$ is that in terms of $\vec{x}_2$ coordinates. We consider a
trial function
\begin{equation}\label{s3e2}
\phi(\vec{x}_1, \vec{x}_2) = \phi_1(\vec{x}_1)\phi_2(\vec{x}_2),
\end{equation}
where
\begin{eqnarray}
\phi_1(\vec{x}_1) &=& \left(\frac{Z^{\prime 3}}{\pi a_0^3}\right)^{1/2}\exp\left(-\frac{Z^\prime r_1}{a_0}\right) \\
\phi_2(\vec{x}_2) &=& \left(\frac{Z^{\prime 3}}{\pi a_0^3}\right)^{1/2}\exp\left(-\frac{Z^\prime r_2}{a_0}\right)
\end{eqnarray}
$Z^\prime$ is the `effective atomic number', a free parameter in our variational
analysis and 
\begin{equation}\label{s3e5}
a_0 = \frac{h^2}{4\pi^2 me^4},
\end{equation}
is the `radius of the first orbit' of the Hydrogen atom. The functions $\phi_1$ 
and $\phi_2$ are eigenfunctions of the hydrogenic hamiltonians with atomic
number $Z^\prime e$. That is,
\begin{eqnarray}
-\frac{\hslash^2}{2m}\nabla_1^2\phi_1 - \frac{Z^\prime e^2}{r_1} &=& -Z^{\prime 2}W_H\phi_1 \label{s3e6} \\
-\frac{\hslash^2}{2m}\nabla_2^2\phi_2 - \frac{Z^\prime e^2}{r_2} &=& -Z^{\prime 2}W_H\phi_2 \label{s3e7}, 
\end{eqnarray}
where 
\begin{equation}\label{s3e8}
W_H = \frac{e^2}{2a_0}.
\end{equation}
Using equations \eqref{s3e1} and \eqref{s3e2} in equation \eqref{s2e1}, we get
\begin{equation}\label{s3e9}
E = \int dV_1 dV_2 \phi_1^\ast\phi_2^\ast\hat{H}\phi_1\phi_2.
\end{equation}
Now, 
\begin{equation}\label{s3e10}
\hat{H}\phi_1\phi_2 = (\hat{H}_1\phi_1)\phi_2 + (\hat{H}_2\phi_2)\phi_1 + (Z^\prime - Z)e^2\left(\frac{1}{r_1} + \frac{1}{r_2}\right)\phi + \frac{e^2}{r_{12}}\phi,
\end{equation}
where $\hat{H}_1$ and $\hat{H}_2$ are hydrogenic hamiltonians with atomic number
$Z^\prime$. Using equations \eqref{s3e6} and \eqref{s3e7} we get
\begin{equation}\label{s3e11}
\hat{H}\phi_1\phi_2 = -2Z^{\prime 2}W_H\phi + (Z^\prime - Z)e^2\left(\frac{1}{r_1} + \frac{1}{r_2}\right)\phi + \frac{e^2}{r_{12}}\phi.
\end{equation}
Substituting \eqref{s3e11} into \eqref{s3e9} we get
\begin{equation}\label{s3e12}
E = -2Z^{\prime 2}W_H + (Z^\prime - Z)e^2\int dV_1dV_2\phi^\ast\left(\frac{1}{r_1} + \frac{1}{r_2}\right)\phi + \int dV_1dV_2\frac{e^2|\phi|^2}{r_{12}}.
\end{equation}
The integral in the second term can be written as
\[
I_2 = 2\int dV_1\frac{|\phi_1|^2}{r_1} = 8\pi\left(\frac{Z^{\prime 3}}{\pi a_0^3}\right)\int_0^\infty\exp\left(-\frac{2Z^\prime r_1}{a_0}\right)r_1 dr_1.
\]
If $\rho = 2Z^\prime r_1/a_0$ then
\[
I_2 = \frac{2Z^\prime}{a_0}\int_0^\infty e^{-\rho}\rho d\rho = \frac{2Z^\prime}{a_0}.
\]
The second term of equation \eqref{s3e12} can now be written as
\[
(Z^\prime - Z)e^2 \times \frac{2Z^\prime}{a_0} = 4(Z^\prime - Z)Z^\prime W_H,
\]
where $W_H$ is given by \eqref{s3e8}. Thus,
\begin{equation}\label{s3e13}
E = -2Z^{\prime 2}{W_H} + 4(Z^\prime - Z)Z^\prime W_H +  \int dV_1dV_2\frac{e^2|\phi|^2}{r_{12}}.
\end{equation}A
Evaluation of the third term is a little challenging. We proceed as follows. 
The integrand of the term is
\[
\frac{e^2}{r_{12}}\frac{Z^{\prime 6}}{\pi^2 a_0^6}\exp\left(-\frac{2Z^\prime}{a_0}(r_1 + r_2)\right).
\]
and the measure of the integral is $r_1^2\sin\vartheta_1dr_1d\varphi_1 d\vartheta_1 r_2^2\sin\vartheta_2dr_2d\varphi_2 d\vartheta_2$.
Introduce new variables $\rho_1 = -2Z^\prime r_1/a_0$ and $\rho_2 = -2Z^\prime r_2/a_0$. Then
\[
|\vec{x}_1 - \vec{x}_2| = |r_1 \vec{e}_{r_1} - r_2 \vec{e}_{r_2}| = \frac{a_0}{2Z^\prime}|\rho_1\vec{e}_{\rho_1} - \rho_2\vec{e}_{\rho_2}|
\]
so that
\begin{equation}\label{s3e14}
\frac{1}{r_{12}} = \frac{2Z^\prime}{a_0}\frac{1}{|\vec{\rho}_1 - \vec{\rho}_2|} = \frac{2Z^\prime}{a_0}\frac{1}{\rho_{12}}.
\end{equation}
The measure of the integral becomes
\begin{equation}\label{s3e15}
dV_1 dV_2 = \frac{a_0^6}{64 Z^{\prime 6}}\rho_1^2\sin\vartheta_1d\rho_1 d\varphi_1 d\vartheta_1\rho_2^2\sin\vartheta_2d\rho_2 d\varphi_2 d\vartheta_2.
\end{equation}
The third term of equation \eqref{s3e13} thus becomes $T_3 = $
\begin{equation}\label{s3e16}
T_3 = \int_0^\infty\int_0^\pi\int_0^{2\pi}\int_0^\infty\int_0^\pi\int_0^{2\pi}\frac{Z^\prime e^2}{32\pi^2 a_0}\frac{e^{-\rho_1 - \rho_2}}{\rho_{12}}\rho_1^2\sin\vartheta_1\rho_2^2\sin\vartheta_2 
d\rho_1d\varphi_1d\vartheta_1 d\rho_2d\varphi_2d\vartheta_2
\end{equation}
The angular integrals can be readily evaluated so that
\begin{equation}\label{s3e17}
T_3 = \frac{Z^\prime e^2}{32\pi^2 a_0}\int_0^\infty\int_0^\infty \frac{(4\pi\rho_1^2e^{-\rho_1}) \times (4\pi\rho_2^2e^{-\rho_2})}{\rho_{12}}d\rho_1 d\rho_2.
\end{equation}
The integral, aside from the constant, is the electrostatic energy of two
spherically symmetric charge distributions of densities $e^{-\rho_1}$ and
$e^{-\rho_2}$. We will, therefore, use tricks from electrostatics to evaluate
it. The potential due to a spherical shell with charge density $4\pi\rho_1^2e^{-\rho_1}$
is
\[
dV(r) = \begin{cases}
4\pi\rho_1^2e^{-\rho_1}\frac{1}{\rho_1} & r < \rho_1 \\
4\pi\rho_1^2e^{-\rho_1}\frac{1}{r}      & r \ge \rho_1
\end{cases}
\]
Therefore, the potential at $r$ due to the entire distribution is
\[
V(r) = \frac{4\pi}{r}\int_0^re^{-\rho_1}\rho_1^2d\rho_1 + 4\pi\int_r^\infty e^{-\rho_1}\rho_1d\rho_1 = \frac{4\pi}{r}(2 - e^{-r}(r + 2)).
\]
Equation \eqref{s3e17} now becomes
\[
T_3 = \frac{Z^\prime e^2}{32\pi^2 a_0}\cdot 4\pi\cdot\int_0^\infty V(\rho_2)e^{-\rho_2}\rho_2^2d\rho_2
 = \frac{Z^\prime e^2}{2a_0}\int_0^\infty [2 - e^{-\rho_2}(\rho_2 +2)]e^{-\rho_2}\rho_2d\rho_2.
\]
The integral evaluates to $5/4$ and hence
\begin{equation}\label{s3e18}
T_3 = \frac{5}{4}\frac{Z^\prime e^2}{2a_0} = \frac{5}{4}Z^\prime W_H,
\end{equation}
where we used \eqref{s3e8} for the definition of $W_H$. Substituting this in
equation \eqref{s3e13} we get
\begin{equation}\label{s3e19}
E = \left[-2Z^{\prime 2} + 4Z^\prime(Z^\prime - Z) + \frac{5}{4}Z^\prime\right]W_H
\end{equation}
The variational procedure requires that we minimize $E$ with respect to $Z^\prime$.
Taking the derivative with respect to $Z^\prime$, we get
\[
\frac{\partial E}{\partial Z^\prime} = \left[-4Z^\prime - 4Z + 8Z^\prime + \frac{5}{4}\right]W_H
\]
At the extremum,
\begin{equation}\label{s3e20}
Z^\prime = Z - \frac{5}{16}.
\end{equation}
The minumum value of $E$ is
\begin{equation}\label{s3e21}
E_{\text{min}} = -2\left(Z - \frac{5}{16}\right)^2W_H.
\end{equation}
For Helium aton $Z = 2$. $W_H$ is always $-13.6$ eV. Therefore, $E_{\text{min}} 
= -77.45625$ eV. The actual energy level is $-79.00515$ eV. Our estimate is
$2\%$ off the actual value.

\section{Upper and lower bounds}\label{s4}
Equation \eqref{s2e4} gives an upper bound for the energy eigenvalue of the ground
state. However, to get a good feel of what $E$ is we also need a lower bound. The 
theory to get the lower bound was available since the 1930s through the work of D.
H. Weinstein. We will describe it in this section. Once again let $\phi$ be a
variational function as defined in section \ref{s2}. Define the quantities
\begin{eqnarray}
E &=& \int\phi^\ast\hat{H}\phi dV \label{s4e1} \\
D &=& \int \left(\hat{H}\phi\right)^\ast \hat{H}\phi dV \label{s4e2}
\end{eqnarray}
If we expand $\phi$ in terms of the eigenfunctions $\psi_n$ of $\hat{H}$ as we
did previously in equation \eqref{s2e2}, we get
\begin{eqnarray}
E &=& \sum_{n \ge 0}|a_n|^2W_n \label{s4e3} \\
D &=& \sum_{n \ge 0}|a_n|^2W_n^2 \label{s4e4}
\end{eqnarray}
Let
\begin{equation}\label{s4e5}
\Delta = D - E^2.
\end{equation}
We manipulate the right hand side of the above equation as
\[
\Delta = D - 2E^2 + E^2 = D - 2EE + E^2 \cdot 1 = \sum_{n \ge 0}|a_n|^2W_n^2 
-2E\sum_{n \ge 0}|a_n|^2W_n + E^2\sum_{n \ge 0}|a_n|^2,
\]
where we have used the fact that $\phi$ is normalized so that
\[
\sum_{n \ge o}|a_n|^2 = 1.
\]
We can now write
\begin{equation}\label{s4e6}
\Delta = \sum_{n \ge 0} |a_n|^2(W_n - E)^2.
\end{equation}
Among all the energy levels in the spectrum of $\hat{H}$ there will be one, say
$W_k$ that will be closest to $E$. Then $(W_k - E)^2 \le (W_n - E)^2$. Therefore,
equation \eqref{s4e6} becomes
\[
\Delta \ge (W_k - E)^2.
\]
Of the two possibilities $W_k \ge E$ and $W_k < E$, the first one gives
\begin{equation}\label{s4e7}
E + \sqrt{\Delta} \ge W_k \ge E
\end{equation}
while the second one gives
\begin{equation}\label{s4e8}
E > W_k > E - \sqrt{\Delta}.
\end{equation}
We can combine equations \eqref{s4e7} and \eqref{s4e8} to get
\[
E - \sqrt{\Delta} \le W_k \le E + \sqrt{\Delta}.
\]
Using the definition of $\Delta$ from equation \eqref{s4e5} we get
\begin{equation}\label{s4e9}
E - \sqrt{D - E^2} \le W_k \le E + \sqrt{D - E^2}.
\end{equation}
This equation gives an interval in which the energy level closest to $E$ lies.
\end{document}
